Structure from Motion (SfM) is a process in which a robot constructs a map of its surroundings and, at the same time, make use of the map to compute its location \cite{forsyth2011computer}. SfM is also one of the hot topic in the field of computer vision research that have provided real world usage for upcoming technologies like autonomous vehicles \cite{kim2013real}. However very little research on how real world variable like hardware limitation, dynamic environments, and the difficulty of capturing ground truth could affects the test data that would be used on SfM. This makes it harder for researchers to test and improve their own interpretation of SfM as it is hard to identify where the problem lies. This paper aims to see how data from a simulated environment helps researchers with testing and improving SfM algorithms. 

In this paper, a Feature-based detectors for SfM - ORB\_SLAM2 \cite{lin2016repmatch} will be used as the algorithm for testing and the paper will be looking at the process of creating an application to generate simulated data. This paper will emphasize on the usefulness of the generated test data in comparision with real world data. Also, a key point in this research is to create an application that could be used as a platform for researchers to generate their own test data in a simulated environment.

I believe that the lack of application for generation of test data for SfM is mainly due to researchers focusing on improving SfM and have no emphasis on potential real world problems that could be derived from hardware limitation, dynamic environment and the difficulty of capturing ground truth. Even though improving on the actual algorithm is vital, we must not forget that in terms of a research environment, having a more controlled enviroment and having controlled data is also desired.

To aid in the creation of the application, I will need to understand how SfM works, data that is required for SfM, and have a set of real world data that could be used in comparision with the application's simulated data. As such, I am working with a researcher in I\textsuperscript{2}R whose key research is in Computer Vision.